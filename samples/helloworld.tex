% Символ % указывает, что текст, следующий за ним до конца
% строки, игнорируется и может использоваться в качестве комментария.

\documentclass{article} % Класс печатного документа.

\usepackage[utf8]{inputenc} % Кодировка исходного текста.
\usepackage[russian]{babel} % Поддержка русского языка.
\usepackage{indentfirst} % Отступ в первом абзаце.
\usepackage{hyperref} % Гиперссылки

\title{Образец текста} % Заголовок документа.
\author{Н.\,Е. Образцов} % Автор документа.
\date{8 июня 2002 года } % Используйте \date{\today}, % чтобы напечатать текущую дату.
\begin{document} % Конец преамбулы, начало текста.

\maketitle % Печатает заголовок, список авторов и дату.

\begin{abstract} % Печатает аннотацию.

Это образец входного файла.
Сравнивая его с готовым печатным документом, нетрудно освоить азы работы с \LaTeX’ом. % Команда \LaTeX печатает логос.

\end{abstract}

\section{Обычный текст} % Печатает заголовок раздела.
% Заголовки подразделов печатают
% аналогичные команды
% \subsection и \subsubsection.
Окончания слов и предложений отмечаются, как обычно,
пробелами. Не имеет значения, сколько пробелов Вы
наберёте; один пробел так же хорош, как и 100.


Одна или несколько пустых строк обозначают конец абзаца.

Поскольку любое количество пробелов рассматривается как один,
способ форматирования текста во входном файле безразличен для
\LaTeX’а.
Однако разумное форматирование входного файла облегчает его
чтение, проверку и внесение изменений.

\subsection{Математические выражения}
\LaTeX\ превосходно печатает как простые математические уравнения типа
\( x-3y = 7 \),
так и более сложные.
Математическую формулу можно записать отдельной строкой:
\[ x’ + y^{2} = z_{i}^{2}. \]
Чтобы пронумеровать формулу, используйте процедуру \texttt{equation}:
\begin{equation}
\int_{-\infty}^{\infty} dx \exp(-x^2) = \sqrt{\pi}.
\end{equation}
\begin{center} % Центрирует текст.
\Large % Команда \Large переключает
% размер шрифта на больший.
Всё остальное Вы узнаете,\\ прочитав \href{https://elitagroup.ru/content/school/web/20170130/Latex_rus.pdf}{эту книгу}.
\end{center}
\end{document} % Конец текста.
